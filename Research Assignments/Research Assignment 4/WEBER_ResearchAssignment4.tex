\documentclass[fleqn,usenatbib]{mnras}

\usepackage{newtxtext,newtxmath}
\usepackage[T1]{fontenc}
\usepackage{graphicx}	
\usepackage{amsmath}	

\title[Short title, max. 45 characters]{Weber Research Assignment 2}

\author{Colin Weber}
\date{4/11/2025}
\begin{document}
\label{firstpage}
\pagerange{\pageref{firstpage}--\pageref{lastpage}}
\maketitle

\begin{keywords}
Major Merger -- Satellite Galaxy -- Dark Matter Halo -- Halo Shape -- Oblate/Prolate/Triaxial

\end{keywords}
\section{Introduction}
        \textbf{Dark matter halos} are believed to be located around galaxies, extending much further than the visible length of their galaxy. A dark matter halo is a large cloud of dark matter that engulfs the visible matter in a galaxy. The length of the halo is typically much larger than the visible length of the galaxy. Since a dark matter halo has a gravitational force, it stands to reason that the dark matter halo of a \textbf{satellite galaxy} would change in shape drastically due to a nearby \textbf{major collision}. A satellite galaxy is a galaxy that is gravitationally bound by a larger galaxy. A major collision is a collision between two galaxies of similar mass.

        The evolution of the \textbf{halo shape} of satellite galaxies during a nearby major merger is very important to understanding how dark matter and visible matter interact gravitationally. The halo shape is the shape of the dark matter halo. This can include where dense regions reside, how stretched out it is, its overall ellipticity, its radius, and its general shape around the rotational axis of the galaxy. The shape around the rotational axis is either \textbf{prolate}, \textbf{oblate}, or \textbf{triaxial}. An oblate shape is a spheroid that is stretched out to resemble the shape of a disk, a prolate shape is a spheroid that is stretched out to resemble the shape of a football, and a triaxial shape is a combination of an oblate and a prolate shape. With the information provided by understanding the shape of the halo, we can better understand \textbf{galaxy evolution} by being able to predict how the gravitational pull of the dark matter halo can reshape the galaxy. Galaxy evolution is how \textbf{galaxies} throughout the universe form, merge, and evolve over time. A galaxy is a gravitationally bound set of stars whose properties cannot be explained by a combination of baryons and newtons laws of gravity. The halo shape can also increase our knowledge of galaxy evolution by allowing us to more accurately determine the gravitational pull of galaxies on one another during a galactic merger. Since so much of a galaxies mass is part of the dark matter halo, it would be nearly impossible to predict how galaxies interact gravitationally without it. 

        At this point, we know for sure that dark matter has a gravitational pull on the mass around it. Otherwise, galaxies wouldn't have enough of a gravitational force to hold themselves together. Dark matter accounts for roughly 85 percent of the mass in the universe. Since dark matter accounts for most of the mass in any given galaxy, it is really important to understanding how galaxies evolve as they collide. We have also observed evidence of dark matter interacting differently than light during galaxy collisions (\citet{Clowe2006}). We know that dark matter is distributed in clumps (\citet{Banik2018}). We know that the gravitational pull of a galaxies dark matter halo can affect the shape of visible objects within a galaxy like stellar clusters (\citet{Erkal2017}). The dark matter halo also provides a medium to slow down galaxies when they pass through each other by causing an energy loss, allowing for them to merge instead of just passing through each other. As shown in \textbf{figure 1}, we also know that the dark matter halo of a galaxy can behave differently than the visible matter during a galaxy merger.
        
\begin{figure}
                \centering
                \includegraphics[width=1.0\linewidth]{Screenshot 2025-03-20 230145.png}
                \label{fig:enter-label}
            \caption{on the left is a contour diagram of the mass distribution of the galaxy measured using gravitational lensing. On the right is an x-ray of the same galaxy. Not only does this serve as a way to map it, but it also confirms that the dark matter behaves differently than the visible mass of the galaxy due to a galaxy collision.}
            \end{figure}
                        
        Many of the primary questions relating to dark matter relate to its distribution. Depending on different models, dark matter is believed to be distributed in either many smaller clumps or fewer large clumps. Dark matter could also be distributed in several different ways around the rotational axis of the galaxy. The dark matter halo around galaxies could be either a prolate, oblate, or triaxial halo. It has been theorized that dark matter clumps could exist as galaxies of their own. Dark matter galaxies could explain the irregular shapes of some galaxies. This could be what is causing the irregular shape of a stellar cluster orbiting around the Milky Way (\citet{Erkal2017}). While we know that it can be affected differently than visible mass during a galaxy merger, we don't know why the dark matter halo can behave differently. Another big unknown relating to dark matter is how it can be transferred from one galaxy to another. We are not sure how the \textbf{Jacobi radius} for the dark matter halo of a satellite galaxy matches what we would expect from an analytical solution of the Jacobi radius equation. The Jacobi radius is the maximum distance a gravitationally object can go until it is stolen by the gravitational pull of another object. How the Jacobi radius evolves as a third body is introduced into the system is also something that remains a mystery. Our inability to directly observe dark matter makes it difficult to attempt to answer these questions. Because we can't directly observe the dark matter halo around each galaxy, we need to observe how the visible matter enclosed within the halo evolves in ways that can't be explained without the existence of dark matter. We can use this to help determine the shape required for the dark matter to influence the visible matter in the way that it does. Since dark matter has a strong gravitational pull, we can also observe the gravitational lensing caused by the galaxy to help determine the shape of the halo.


\section {proposal}
\subsection{This Proposal}
        I am aiming to analyze the shape of the dark matter halo around the Triangulum galaxy (M33) at several key points during the merger of the Milky Way and Andromeda (M31) galaxies to see how it evolves over time. M33 is a satellite galaxy of the Milky Way, meaning that it will be greatly affected by the merger between M31 and the Milky Way. The main focuses that I am going to be analyzing are the shape around its orbital axis (oblate, prolate, triaxial), how the halo appears to change in physical shape, and how the Jacobi radius of the Triangulum galaxy evolves as the merger progresses and M33 reaches its closest and furthest points from the merger.

        This project aims to address the shape of the halo about its orbital axis. It also aims to address how the analytical Jacobi radius compares to the simulated radius and how both of them evolve over time as the distances of the Milky Way, M31, and M33 change during the merger.


        This project will help us understand how the mass of galaxies evolves over time. It also helps us understand how a dark matter halo behaves when the nearby mass suddenly increases by a large amount. It will help us understand the behavior of the halo as it reaches its pericenter and apocenter. The shape of the halo could have large effects on the gravitational pull between M33 and the merged Milky Way and M31 galaxy.



\section{Methodology}
        The simulations being used are the ones that were used ((\citet{vanderMarel2012})). Here an \textbf{N-body} simulation of the Milky Way, M31, and M33 system was modeled. An N-body simulation is a simulation of how 3 or more gravitationally bound objects with initial masses and velocities evolve with time.

        I am going to be using the dark matter particles from the M33 galaxy in the simulation data. At this point I believe that it will be best to use the HighRes simulation data to get the most accurate data possible. I am planning on using several different snapshots. I am planning on looking at the 000, 300, 340, 380, 705, and 740 snapshots. Snap 000 is the start of the simulation, so it is very important to understand its shape in order to see how it evolves over time. Snaps 300, 340, and 380 were chosen because of how close they are in time to the galaxy merger. Snap 300 is a point where M31 is at pericenter. In snap 380, M33 is at apocenter. point 340 was chosen because it is a point in between them. Snap 705 is an apocenter snap and snap 740 was a pericenter snap. These were chosen because they take place a significant amount of time after the merger.


        The first thing my code will compute is the distance between the M31 and M33 galaxies and identifying the apocenters, pericenters, and midpoints of the distances between M31 and M33. My code will also compute the rotational axes that the dark matter halo rotates around, then it will redefine each particle in terms of this new coordinate system. My code will be computing how the mass enclosed within a cylinder along each axis changes as the radius of the cylinder increases. The Jacobi radius can be calculated using the equation $r_j = d(M_1/{2M_{tot}})^{1/3}$. (Szebehely 1967) where $r_j$ represents the Jacobi radius, $d$ represents the distance between the host and satellite galaxies, $M_1$ is the mass of the satellite galaxy, and $M_2$ is the total mass of the host galaxy enclosed within a radius $d$

\begin{figure}
                \centering
                \includegraphics[width=0.7\linewidth]{Fig2.jpg}
                \label{fig:enter-label}
            \caption{ Two cylinders of equal radius, r, and a length of infinity surround a prolate shape. The cylinder along the y axis clearly contains more of the object than the cylinders along the x and z axes. By using this, we can determine if the shape is prolate, oblate(two cylinders contain more mass in r than the third), or triaxial (differing masses for each cylinder)}
            \end{figure}
        To analyze whether or not the galaxy is oblate, prolate, or triaxial, I am planning to run a function that computes the total mass enclosed within a cylinder along each of the 3 major axes. As presented in \textbf{figure 2}, In theory, the graph of mass enclosed for the cylinder will reach a certain threshold of total mass enclosed in one direction much more quickly than the other two if it is prolate. If two cylinders reach the threshold much more quickly than the third direction, it is likely oblate. If they all reach the threshold in very different radii, then it is likely triaxial. To measure the Jacobi radius, I will be making a mass density profile of M33 and plotting the analytical solution of the Jacobi radius over it. Finally, I will also be creating contour diagrams that are both face on and edge on for the dark matter halo of M33 at each of the snapshots that I am going to be using. This lets me see how the halo changes in shape as the merger goes on. Specifically, this will tell me if the halo appears to be stretched out as the simulation goes on.

        I believe that the halo should be prolate, and that the Jacobi radius will vary wildly at different points of the simulation. I expect the halo to be prolate due to the gravitational pull of M31 and the Milky Way on M33. I believe that this will cause the halo to become elongated in a way that points towards the center of mass of the Milky Way galaxy and M31 galaxy. The Jacobi radius should change drastically as the simulation progresses. As seen in the equation of the Jacobi radius, it depends primarily on the mass and distance of the objects. The distance between M33 and M31 oscillates throughout the simulation, meaning that the Jacobi radius will change with it. During the simulation, M31 merges with the Milky Way, meaning that the total mass of the galaxy that M33 is orbiting around is very different by the end of the simulation. This should cause the Jacobi radius to be much smaller at the end of the simulation than the beginning of the simulation, assuming a similar distance between M31 and M33 at both time snaps. Since M33 is suddenly going to be around considerably more mass due to the merger between M33 and the Milky Way, I expect to see the halo be more stretched out in the direction facing M31.



\bibliographystyle{mnras}
\bibliography{references} 
\cite{}
\label{lastpage}
\end{document}